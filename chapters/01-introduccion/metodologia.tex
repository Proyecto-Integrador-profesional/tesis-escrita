\section{Metodología}

En primer lugar se diseña el sistema permitiendo obtener los requisitos del sistema. A grandes rasgos, se pueden diferenciar tres tareas principales: implementar un algoritmo de detección óptica de caracteres para obtener la patente de los vehículos, diseñar e implementar un sistema capaz de capturar la fotografía para posteriormente procesarla, y desarrollar una aplicación WEB para permitir el acceso a la información a los dueños del establecimiento.

El desarrollo de la obtención de la patente de un vehículo a partir de una fotografía. La conversión fotografía-patente se realiza utilizando un algoritmo de detección óptica de caracteres usando redes neuronales convoluciones. Posteriormente se desarrolla una implementación del algoritmo en Python 3, para poder utilizarlo en un servidor web, y en sistemas embebidos.

El diseño de las placas secundarias se lleva a cabo con Raspberry Pi 3B+ y NVIDIA Jetson TX1, con la idea de desarrollar dos versiones del sistema, los cuales son denominados como SL mini y SL respectivamente. En esta etapa se definen los sensores necesarios para la tarea de obtener la fotografía de los vehículos. Además se hacen las implementaciones de software necesarias para realizar la tarea.

Para el desarrollo del servidor se eligió una variedad de tecnologías, principalmente escritas en Python 3 y JavaScript, ya que son muy utilizadas en la actualidad. Esta etapa incluye el diseño de un panel para los dueños de los establecimientos, permitiendo acceder a la información y cambiar algunas configuraciones básicas de los sistemas SL. Además los administradores cuentan con un panel específico para administrar los sistemas SL, permitiendo crear nuevos sistemas.

Para estudiar el rendimiento del sistema se realizan diferentes pruebas tales como: prueba de exterior, distancia y ángulos variables y creación de registros de entrada y salida. La prueba en exterior permite ver el rendimiento del algoritmo de detección óptica de caracteres en un ambiente para el cual no fue diseñado. El análisis de distancia y ángulos hace énfasis en el comportamiento del sistema en entornos controlados, y se realiza con la finalidad de estudiar los límites del algoritmo de detección óptica de caracteres. Por última la prueba sobre el servidor, permite observar si el comportamiento de la aplicación web es correcta, permitiendo crear registros y visualizar los datos de forma correcta.

