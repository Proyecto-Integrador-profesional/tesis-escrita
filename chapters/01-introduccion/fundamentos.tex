\section{Fundamentos}

Debido a la creciente flota de vehículos circulantes a nivel mundial y principalmente en Argentina la cual tuvo un crecimiento del 1,88\% entre 2020 y 2021 \cite{asociacion_de_fabricas_argentinas_de_componentes_flota_2021}, logrando un total de 14.840.010 a finales del 2021. Particularmente en la ciudad de Neuquén, la problemática para conseguir zonas de estacionamiento se ve amplificada por la constante llegada de vehículos de ciudades aledañas principalmente por razones de estudio o trabajo, llegando casi a duplicar la flota de vehículos circulantes en el día a día \cite{noauthor_imposible_nodate}. Este problema no es único de esta ciudad, a lo largo de todo el mundo ya se han encontrado problemas similares \cite{20minutos_falta_2018, ibrahim_car_2017}.

Una problemática similar que afectó a las industrias a lo largo del mundo, las llevó a la búsqueda de nuevas soluciones. Gracias a los avances  en diversas tecnologías tales como: robótica, simulación, sistemas de integración, IOT (Internet of things), IA (Inteligencia artificial), Ciberseguridad, Big data, entre otras, la industria llegó a una nueva era: Industria 4.0, donde la tecnología es cada vez más utilizada para realización de tareas y resolución de problemas buscando una industria autónoma \cite{basco_industria_2018}.

Muchas de estas ideas de industrias inteligentes, comenzaron a ser extrapoladas a otros aspectos de la vida cotidiana, dando paso a la idea de ciudades inteligentes. Esto se debe principalmente a los aportes en ahorro de recursos que pueden brindar. Dentro de los servicios fundamentales que debe afrontar una ciudad inteligente están: eficiencia energética y medioambiente, gestión de infraestructuras y edificios públicos, seguridad pública y movilidad urbana, siendo esta última  donde se busca generar un aporte, y una base para futuros trabajos. La idea general es integrar nuevas tecnologías, a una idea antigua, es decir, modificar el sistema de control de estacionamiento clásico, buscando la automatización. Este avance presenta mejoras tanto para los usuarios del sistema, que desean estacionar y pueden hacerlo de una forma más eficiente, gracias a la eliminación de intermediarios. Por otro lado, al dueño o administrador, ya que le permite llevar un control de rendimiento mucho más sencillo, permitiendo planificar de una mejor manera sus inversiones, dando datos estadísticos de uso reales de sus usuarios, como así también permitiendo generar promociones a clientes habituales basados en reglas simples, como horas semanales.

Esta idea de un estacionamiento inteligente, está siendo explorada desde otros enfoques en diversos países a lo largo del mundo, por ejemplo control de espacios por sensores de ultrasonido \cite{rivera_arroyave_smartparkudea_2021}, plataforma inteligente de estacionamiento \cite{formoso_parkit_2014}, entre otros. Nuestro enfoque en cambio se centra en la visión artificial. Mediante la detección y reconocimiento de caracteres de imágenes tomadas desde una cámara, se busca reconocer y almacenar la patente de cada vehículo que ingresa y egresa del espacio destinado al estacionamiento.
