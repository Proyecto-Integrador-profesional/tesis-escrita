\chapter{Implementación de hardware}
En este capítulo se detallara cual es el hardware utilizado para el desarrollo del sistema de acceso, se indicarán cuales
son los requisitos mínimos que deben satisfacerse para que el sistema cumpla las características solicitadas, como
también se indicarán los sensores y equipos auxiliares usados junta con otras posibles opciones encontradas y las
razones por las que estas fueron descartadas.


\section{Estado actual del sistema de acceso por barreras}

\subsection{Sistemas tradicionales}

Hoy en dia el sistema de uso de barreras para el control de ingreso y egreso a distintos resintos suelen generar molestias
en muchos de los usuarios, por la necesidad de realizar alguna accion extra, con esto nos referimos a la necesidad de 
en muchas ocaciones de obtener y guardar algun tipo de ticket que en caso de perderlo se tenga que pagar una
multa economica, o bien de acercarse a un determinado lugar para poder realizar el pago por el tiempo de permanencia.

En este tipo de metodologias es donde queremos realizar un aporte a la reduccion del impacto ecologico, ya que si se quita
la necesidad de imprimir uno o mas tickets por cada vehiculo que ingresa, se estaria disminuyendo considerablemente la 
necesidad de utilizar papel y considerando que el metodo de impresion de estos tickets suele ser por impresion termica,
proceso donde se utiliza un papel termosensible que al ser calentado se vuelve negro, que tiene un costo electrico
adicional.

Otro de los aspectos que destacan del uso actual del sistema de barreras en sistemas mas manuales es la necesidad de 
contar con  operarios trabajando en la barrera el tiempo que la barrera este disponible, ya que en caso de no disponerlo 
debera quedar la barrera sin efecto, perdiendo por completo su utilidad.

\subsection{Sistemas Modernos}

Con el avance y el abaratamiento de los costos en la electronica surgieron nuevos metodos que permitieron a los usuarios
precindir de 
 
Otro sistema que esta ocupando gran parte del mercado en los ultimos años es el que integra a la barrera un sistema de 
RFID, que mediante la colocacion de un emisor RF en el vehiculo y un receptor en la barrera, al acercarse al ingreso
se produce en el enlace que habilita o no al vehiculo a ingresar.

\section{Selección de placas}

\section{Evalución y selección de sensores}

\section{Consumo energetico}

\section{Diseño y ensamble}