\chapter{Conclusiones}
En este capítulo se realizarán las conclusiones generales correspondientes al uso de redes neuronales convoluciones para el uso de detección de caracteres, más específicamente para detectar patentes argentinas, se detallarán las ventajas y desventajas encontradas en los algoritmos, para finalmente presentar posibles mejoras a futuro con el objetivo de mejorar diferentes aspectos del prototipo.

\section{Conclusiones generales}

Durante el proceso de investigación sobre OCR y sus utilidades, se observó que en nuestro país este tipo de tecnologías son poco explotadas, con aplicaciones escasas, limitándose al uso de herramientas creadas por terceros.
Por lo que el diseño de este prototipo representa una innovación en lo que a control de acceso de estacionamientos respecta, permitiendo una automatización y control del acceso de los vehículos al recinto con una interacción nula por parte del usuario.

Se han estudiado en profundidad el funcionamiento de redes neuronales, más específicamente las CNN y se observó que estas son las más favorables para la implementación en sistemas embebidos, por su bajo costo computacional en comparación con las LSTM, logrando reconocer las patentes de manera efectiva.

Uno de los enfoques más importantes de este trabajo fue el diseño modular, buscando que la interdependencia de las partes del sistema sea lo más baja posible, es decir, permite cambiar partes del sistema disminuyendo los errores al integrar estos nuevos elementos.
Por ejemplo, es posible cambiar el servicio de detección de patentes por uno mejor, sin necesidad de cambiar el código del Backend o del Frontend solo se debe respetar que la comunicación entre el Backend y este servicio sea por HTTP y devuelva el JSON esperado.
A su vez este diseño permite cambiar el sensor de distancia o la cámara sin tener un gran impacto en el resto del sistema.
En síntesis este tipo de diseño permite una mayor flexibilidad a la hora de integrar nuevas tecnologías.

En general las tareas de diseño y desarrollo son complejas, sobre todo a la hora de brindar soluciones innovadoras.
Sin embargo se logró desarrollar un prototipo funcional para permitir el acceso de vehículos a un estacionamiento sin interacción de los usuarios.

\section{Posibles mejoras}
Finalmente se presentan posibles mejoras en los sistemas de obtención de imágenes, detección y obtención de caracteres estudiados a lo largo de este trabajo:
\begin{itemize}
    \item Implementar un nuevo algoritmo para la detección de patentes en otro tipo de vehículos (motos, camiones).
    \item Mejorar la eficiencia del algoritmo de OCR.
    \item Desarrollar e implementar de algoritmo de reconocimiento de patentes extranjeras.
    \item Reducir el costo de ensamble de los sistemas SL.
    \item Sustituir la cámara por una que permita mejores fotografías en ambientes externos.
    \item Integrar una pasarela de pagos para efectuar el cobro del estacionamiento.
    \item Desarrollar una versión con display para mejorar la experiencia de los usuarios.
\end{itemize}
