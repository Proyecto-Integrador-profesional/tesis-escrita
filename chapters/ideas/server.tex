\section{Server}

En este apartado se busca hablar sobre el servidor, su arquitectura basada en microservicios usando docker, y la base de datos.

\subsection{Microservicios}


Que es un microservicio, que microservicios tiene el servidor (base de datos, servidor web en nestjs), aca se habla de docker y el uso de docker compose, con la posibilidad de migrar a kubernetes si el sistema crece. [se van agregar explicaciones y documentacion necesaria para el
                entendimiento de dichas tecnologias] \\

El servidor cuenta con los siguientes microservicios:

\begin{enumerate}
        \item Base de datos
        \item Applicacion backend con nestjs
              \subitem Usuarios
              \subitem Barreras
              \subitem Parques de estacionamiento
              \subitem Registros
        \item Frontend con ReactJS
        \item Applicacion backend con Flask (analisis de patente online)
        \item [no implementada] Grafana para usuarios
\end{enumerate}

\subsection{Frontend vs Backend}

comparacion de los servicios frontend y backend explicando sus diferencias \\

\subsubsection*{Backend con NestJS}


Para la autenticación se utilizo el module de Passport JS, el cual permite crear estrategias para los diferentes tipos de autenticación, las estrategias utilizadas son:

\begin{enumerate}
        \item Local
        \item Apikey aplicaciones de terceros
        \item Apikey para las barreras
        \item JWT
\end{enumerate}

La estrategia Local, hace un logging del usuario segun su nombre de usuario y su contraseña.

Los servicios de apikey buscan que la apikey sea valida, es decir, se encuentre en la base de datos vinculada a una barrera, o este dentro de las apikeys validas de aplicaciones, en este caso solo 2 servicios se encuentran habilitados para realizar peticiones (Frontend y Grafana)

Un JSON Web Token es un estandar qué esta dentro del documento RFC 7519, este define un mecanismo para enviar información entre 2 partes, en este caso cliente-servidor de forma segura (utilizando cifrado). Los datos son codificados en un objeto de javascript (JSON), 

Se tratara en mayor profundidad sobre el servicio web del backend, nombrando endpoints y tecnologias utilizadas, describiendo las funcionalidades y su por qué.


Base de datos [mongodb] \\

explicacion de base de datos relacional y no relacional, por qué elegimos mongoDB para el almacenamiento de datos. Y qué es lo que se guarda. Tambien se explicara como se guardan las imagenes. \\
