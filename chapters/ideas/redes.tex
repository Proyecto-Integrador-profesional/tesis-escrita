\section{Redes}

\subsection{Introducción a las redes}

Como se comentó uno los ejes de este trabajo se va a basar en la utilización de redes neuronales para la detección y reconocimiento de patentes y sus correspondientes caracteres, entonces ¿Qué es una red neuronal?, a grandes rasgos una red neuronal en un algoritmo computacional que intenta imitar el cerebro humano, en otras palabras es un sistema de computación compuesto por un gran numero de elementos simples que se encuentran interconectados, los cuales procesan la información por medio de estados dinámicos, respondiendo a entradas externas.[acá iría referencia]

Debido a la semenjanza que existe entre los algoritmos de deep learning y el cerebro humano, a la unidad fundamental de estos sistemas se la denomina neurona.
Cada neurona procesa la información de la capa de anterior y la entrega a la siguiente capa. Este proceso se puede entender como la sinapsis neuronal de los seres vivos.
La sinamsis es una suma poderada que puede ser expresada como $Y = W^T\dot X + b$ donde $W$ representa los pesos de la neurona, $b$ es un bias y $X$ es el vector de entrada.

Una capa es un arreglo en paralelo de neuronas, las capas se interconectan para crear redes más complejas capaces de realizar tareas más especificas [agregar figura de una red y sus partes].
Debido a que en esencia el proceso que realiza la neurona es una transformación lineal, al interconectar capas la resultante sigue siendo una transformación lineal. Este problema de linealidad se soluciona apilcando una función de activación luego de la transformación lineal, obteniendo $Y=f(W^T \dot X + b)$. [agregar grafico de funciones de activación].

Para lo que a procesamiento de imagen y extracción de características las redes mas utilizadas son las CNN , ya que su diseño se vaso en la estructura de la corteza visual animal, esta imitación se consigue utilizando la convolución en 2 dimensiones.
La convolución es 2 dimensiones es similar al caso conocido en  1 dimensión con algunas modificaciones, por ejemplo se habla de sistemas LSI (lineales de espacio invariante) en vez de LTI, para tenerla presente se procede a explicar la definición de convolución discreta en 2 dimensiones
Se define filtro o núcleo  o matriz de convolución a la respuesta de un sistema LSI discreto, el filtro es dimensión $2k x 2k$ donde $k$ es un valor establecido arbitrario (usualmente son matrices cuadradas de $3x3$ o $5x5$) que define cuantos valores habra de la muestra.
Se define a $h[n]$ como un filtro de dimension $2k x 2k$ e $I$ una imagen a escala de grises, donde cada punto de coordenadas $(i,j)$ es el resultado de la convolucion entre $h$ e $I$ dado por 
\begin{equation}
O(i,j)= \sum_{u=-k}^{k} \sum_{v=-k}^{k} h(u,v)I(i-u,j-v)
\end{equation}
 
Esta operacion consiste en filtrar una imagen de dimension $(2k+1)x(2k+1)$ en la imagen $I$ y para cada pixel centrado en dicha imagen, calculando la operacion de convolucion.
Es necesario definir algunos conceptos antes de continuar:
\begin{itemize}
\item Paso $S_{w,h}$ distancia en pixeles que se da entre aplicacones sucesivas de la convolucion.
\item Relleno cero $P_{w,h}$ numero de ceros que se deben añadir como borde al resultado de la convolucion.
\end{itemize} 
(revisar si esto queda o no)

El ajustes de los filtros necesarios para extraer caracteristicas puede resultar un proceso largo y tedioso que no siempre conduce a los resultados que uno esperaria, por lo que se puede utilizar machine learning para obtener los valores de los filtros, para que se pueden extrar de manera satisfactoria las caracteristicas de las imagenes.
Para esta tarea, al colocar la neuronas en cuadriculas de $2k x 2k$, pueden usarse como filtros de convolucion para la extracion de caracteristicas y modificar sus valores durante la propagacion hacia atras (back propagation, metodo de entrenamiento que sera explicado mas adelante)(ver tema de filtros en paralelo F_d profunidad de filtro hiperparametro)


HABLAR DE TIPOS DE NEURONAS CONVOLUCIONALES.

FALTA EXPLICAR EL ALGORITMO DE ENTRENAMIENTO MEDIANTE DATASETS Y BACKPROPAGATION.

Existen diversos tipos clasificación aunque la más utilizada es por el rol que la red puede desempeñar, existen redes reunorales convolucionales o CNN por sus siglas en inglés (comúnmente utilizadas para clasificación de imagenes),
redes neuronales recurrenter o RNN (utilizadas para predicción de texto de largo variable), entre otras. Otra clasificación que es posible utilizal es según la cantidad de capas que esta posea, existiendo 2 clasificaciones, simple las cuales no poseen capas ocultas y profundas las cuales si poseen capas ocultas.

A continuación se dará un breve resumen de como se crean estas capas de neuronas, para ello vamos explicar de manera sencilla y concisa como funciona una red neuronal:
\begin{itemize}
    \item Una capa recibe valores, llamados inputs,si se trata de la primera capa esos valores vendrán definidos por los datos de entrada, mientras que el resto de capas recibirán el resultado de la capa anterior.
    \item Luego se realiza una suma ponderada de todos los valores provenientes de la entrada, para ellos se necesita una matriz de pesos llamada W, la matriz tiene filas igual al numero de de capas anterior y columnas igual como neuronas tiene la capa actual.
    \item Al resultado de la suma ponderada se le sumará otro parámetro, conocido como bias o, simplemente, b. Cada neurona tiene su propio bias, por lo que las dimensiones del vector bias depende de la capa, por lo que será una columna y tantas filas como neuronas tiene esa capa.
    \item Luego se requiere de una función de activación, uno de los elementos mas importantes dentro de la red. Ya que hasta ahora solo tenemos una regresión lineal. Para evitar esto, al resultado de la suma anterior se le aplica una función, conocido como función de activación. El resultado de esta función será el resultado de la neurona.
\end{itemize}

Una vez que se realiza la suma ponderada, lo que tenemos es básicamente una transformación lineal, por lo que para evitar esto, se introduce una función que modifica estos datos, conocida como función de activación, que ademas de sacar linealidad a la red, le permite resolver problemas mas complejos, algunas de las mas usadas son la funcion escalon, la funcion sigmoide, la funcion rectificadora ReLU y la funcion tangete hiperbolica.[poner graficos]


vamos a hablar sobre redes neuronales basadas en imágenes (CNN) y sobre yolo v4.

nuestro algoritmo de entrada/salida \\

explicar como van a funcionar los sensores SL y SL mini a nivel de control de entrada salida,
como lo vamos a hacer, es decir hablar de las 2 redes que vamos a usar. \\



explicación del algoritmo paso a paso \\

se pretende explicar como funciona la cada parte de nuestro algoritmo, y que hace cada una de las redes, mostrando que la red que reconoce los caracteres no lo hace de forma consistente si la imagen no es recortada previamente. Hablamos sobre la yolo v4 y sus ventajas junto con su arquitectura.

