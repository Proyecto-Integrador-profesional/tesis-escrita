\chapter{Ideas de la tesis}

luego del capitulo 1 (plan de pip)\\

vamos a hablar sobre redes neuronales basadas en imagenes (CNN) y sobre yolo v4.

nuestro algoritmo de entrada/salida \\

explicar como van a funcionar los sensores SL y SL mini a nivel de control de entrada salida,
como lo vamos a hacer, es decir hablar de las 2 redes que vamos a usar. \\

introduccion a las redes neuronales \\
desde que es una neurona, que es una capa, aca van tipos capas de las que usamos, y como se conectan. Tambien hablaremos sobre las funciones de activacion y como se entrena a las mismas. \\

explicacion del algoritmo paso a paso \\

se pretende explicar como funciona la cada parte de nuestro algoritmo, y que hace cada una de las redes, mostrando que la red que reconoce los caracteres no lo hace de forma consistente si la imagen no es recortada previamente. Hablamos sobre la yolo v4 y sus ventajas junto con su arquitectura.

comunicacion con el servidor \\

aca se habla sobre la comunicacion via wifi con el server para la persistencia de datos \\

EL SERVER \\

En este apartado se busca hablar sobre el servidor, su arquitectura basada en microservicios usando docker, y la base de datos. \\

microservicios \\

que es un microservicio, que microservicios tiene el servidor (base de datos, servidor web en nestjs), aca se habla de docker y el uso de docker compose, con la posibilidad de migrar a kubernetes si el sistema crece. [se van agregar explicaciones y documentacion necesaria para el 
entendimiento de dichas tecnologias] \\

servi


