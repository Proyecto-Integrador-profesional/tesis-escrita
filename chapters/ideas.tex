\chapter{Ideas de la tesis}

\section{Notas para tener en cuenta}

\begin{enumerate}
    \item Lo relacionado con el algoritmo (redes neuronales) y servidor seran los capitulos con mayor densidad de teoria, debido a que no se vio mucho en la carrera.
    \item En la explicación paso a paso se pretende incluir un diagra de bloques del algoritmo, e ir mostrando que hace cada bloque desde un enfoque agnostico al lenguaje, pero recalcando que se utilizo python por su simpleza a la hora de trabajar con redes.
    \item Lo circundante a Docker será mayormente documentación, aunque se explicara la idea y por qué es algo tan utilizado a nivel de industria.
    \item En la implementación fisica de los sensores, se pretende abarcar el montaje, junto con los planos para lograr ensamblar los sensores.
\end{enumerate}

\section{Un repaso rapido por la PIP}

luego del capitulo 1 (plan de pip)\\

vamos a hablar sobre redes neuronales basadas en imagenes (CNN) y sobre yolo v4.

nuestro algoritmo de entrada/salida \\

explicar como van a funcionar los sensores SL y SL mini a nivel de control de entrada salida,
como lo vamos a hacer, es decir hablar de las 2 redes que vamos a usar. \\

introduccion a las redes neuronales \\
desde que es una neurona, que es una capa, aca van tipos capas de las que usamos, y como se conectan. Tambien hablaremos sobre las funciones de activacion y como se entrena a las mismas. \\

explicacion del algoritmo paso a paso \\

se pretende explicar como funciona la cada parte de nuestro algoritmo, y que hace cada una de las redes, mostrando que la red que reconoce los caracteres no lo hace de forma consistente si la imagen no es recortada previamente. Hablamos sobre la yolo v4 y sus ventajas junto con su arquitectura.

comunicacion con el servidor \\

aca se habla sobre la comunicacion via wifi con el server para la persistencia de datos \\

EL SERVER \\

En este apartado se busca hablar sobre el servidor, su arquitectura basada en microservicios usando docker, y la base de datos. \\

microservicios \\

que es un microservicio, que microservicios tiene el servidor (base de datos, servidor web en nestjs), aca se habla de docker y el uso de docker compose, con la posibilidad de migrar a kubernetes si el sistema crece. [se van agregar explicaciones y documentacion necesaria para el 
entendimiento de dichas tecnologias] \\

frontend vs backend \\

comparacion de los servicios frontend y backend explicando sus diferencias \\

servidor con nestjs \\

Se tratara en mayor profundidad sobre el servicio web del backend, nombrando endpoints y tecnologias utilizadas, describiendo las funcionalidades y su por qué. \\

Base de datos [mongodb] \\

explicacion de base de datos relacional y no relacional, por qué elegimos mongoDB para el almacenamiento de datos. Y qué es lo que se guarda. Tambien se explicara como se guardan las imagenes. \\

EL MONTAJE REAL \\

Se pretende mostrar como esta montado el sensor, hablar sobre las decisiones de diseño, tanto 
camara, como alimentación. Tambien se pretende incluir un analisis de consumo. y resultados de las pruebas, mostrando tanto datos, como imagenes de los resultados. En este apartado la teoria se basara mas que nada en alimentacion y detalles sombre las camaras. \\





