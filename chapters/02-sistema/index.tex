\chapter{Diseño del sistema}

 {\huge se utiliza la palabra barrera para hablar de los sensores SL y SL mini, aun no encuentro una palabra mejor para definirlas sin hacer referencia a eso}

En este capítulo se muestra el desarrollo general del sistema, comenzando por una vision superficial del sistema hasta obtener la versión final del sistema, la implementación del mismo se hará en los capítulos siguientes. [puede ser que se incluya una versión final del sistema en una figura]
\section{Descripción general del sistema}

El requisito fundamental de este trabajo fue poder controlar el sistema de entrada y salida de vehiculos de al menos un recinto. Por lo que nos enfocaremos en comprender como funciona el sistema para el caso de un solo lugar. Adicionalmente existen otros requisitos preestablecidos, como:

\begin{itemize}
    \item Controlar el tiempo que el vehículo estuvo en el establecimiento.
    \item Que la entrada y salida sea automática basada en la patente.
\end{itemize}


\begin{figure}
    \centering
    \includegraphics[width=.8\textwidth]{imgs/sistema-base.png}
    \caption{Versión inicial del sistema}
    \label{fig:sistema-base}
\end{figure}

De estos requisitos se desprende la forma más general del sistema, la cual cuenta con sistema que detecte los caracteres del vehículo, le permita el acceso o el egreso del establecimiento al vehículo y basado en la hora de entra y salida calcule el tiempo que estuvo el vehículo Fig. \ref{fig:sistema-base}.


Pensando en parques de estacionamiento, como los de los supermercados, es usual encontrarse que tienen entradas y salidas diferenciadas, es por ello que se decidió diferenciar la existencia de barreras que actuaran como entrada o salida.

Otra problemática a la hora de pensar en un sistema automático, es ¿Cómo podemos detectar la patente? La respuesta, aunque simple muy utilizada, es implementar una cámara para capturar una foto del vehículo. Un inconveniente es cuando sacar la foto, es por ello que se agregó un sensor para medir distancia, y evitar un período de muestreo estático. De esta manera obtenemos la versión general del sistema Fig. \ref{fig:sistema-completa}.

\begin{figure}
    \centering
    \includegraphics[width=.3\textwidth]{imgs/sistema-con-sensor.png}
    \caption{Sistema con sensor de distancia y cámara.}
    \label{fig:sistema-completa}
\end{figure}

\section{Los algoritmos de entrada y salida}

Una vez establecido nuestro sistema, obtenemos dos algoritmos muy similares, uno para la entrada del vehículo y otro para la salida. Es por ello que comenzaremos describiendo el algoritmo de entrada y luego pasaremos al algoritmo de salida.

\subsection{Algoritmo de entrada}

La toma de una muestras, es decir, una fotografía, se da cuando el sistema detecta que tiene un objeto a menos de $X$cm. Luego, se procesa la imagen, para obtener la patente en formato de texto, en caso de encontrar una patente, se almacena la fecha, junto con la patente y la fotografía Fig. \ref{fig:flujo-entrada}.

\begin{figure}
    \centering
    \includegraphics[width=.5\textwidth]{imgs/flujo-entrada.png}
    \caption{Diagrama de flujo para la entrada}
    \label{fig:flujo-entrada}
\end{figure}

\subsection{Algoritmo de salida}

El algoritmo de salida es idéntico, en la toma de muestra, pero antes de guardar la información de salida (foto y fecha), se verifica que exista un registro de entrada correspondiente y se actualiza el mismo Fig. \ref{fig:flujo-salida}.

\begin{figure}
    \centering
    \includegraphics[width=.5\textwidth]{imgs/flujo-salida.png}
    \caption{Diagrama de flujo para la salida}
    \label{fig:flujo-salida}
\end{figure}

\subsection{El problema de múltiples establecimientos}

Ambos algoritmo poseen un pequeño inconveniente a la hora de extender el proyecto en más de un recinto. Es por ello que cada barrera se vincula a un lugar, y los registros se buscan y actualizan en función del lugar que tiene asignado cada barrera.

\section{Partes del sistema}

 {\huge quizas tengamos que agregar una seccion antes con la teoria de protocolo HTTP}

Una vez diseñado y comprendido como está compuesto el sistema a grandes rasgos, es importante entender las partes del mismo y lograr diferencias que parte del algoritmo se ejecutara en la barrera y cuál en el servidor. Es importante destacar que la comunicación entre el servidor y las barreras se dará por 2 protocolos HTTP y MQTT.

El envío de registros, tanto entrada como salida, es mediante HTTP, por otro lado las configuraciones de las barreras se hacen mediante MQTT, lo que permite que el servidor envíe los cambios en tiempo real, sin que la barrera tenga que realizar una petición por HTTP para cambiar o validar su configuración actual.

El esquema del sistema final, contemplando las barreras y el servidor se puede observar en Fig. \ref{fig:sistema-server-barrera}.

\begin{figure}[h]
    \centering
    \includegraphics[width=.4\textwidth]{imgs/sistema-server-barrera}
    \caption{Sistema separado en barrera y servidor.}
    \label{fig:sistema-server-barrera}
\end{figure}

Esta versión del sistema se desarrollará en los capítulos siguientes, yendo desde la transformación de foto a patente mediante detección óptica de caracteres, pasando por el diseño e implementación del sistema SL, y el diseño e implementación del servidor.